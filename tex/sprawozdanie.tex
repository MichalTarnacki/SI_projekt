\documentclass{article}

\title{Projekt z SI}
\author{Dominik Lau, Mateusz Kowalczyk, Michał Tarnacki}

\usepackage{blindtext}
\usepackage{amsmath}
\usepackage[utf8]{inputenc}
\usepackage[polish]{babel}
\usepackage[T1]{fontenc}
\usepackage{listings}
\usepackage{color}
\usepackage{amssymb}
\usepackage{esvect}
\usepackage{graphicx}

\graphicspath{ {./obrazy/} }

\definecolor{dkgreen}{rgb}{0,0.6,0}
\definecolor{gray}{rgb}{0.5,0.5,0.5}
\definecolor{mauve}{rgb}{0.58,0,0.82}

\lstset{frame=tb,
  language=Python,
  aboveskip=3mm,
  belowskip=3mm,
  showstringspaces=false,
  columns=flexible,
  basicstyle={\small\ttfamily},
  numbers=none,
  numberstyle=\tiny\color{gray},
  keywordstyle=\color{blue},
  commentstyle=\color{dkgreen},
  stringstyle=\color{mauve},
  breaklines=true,
  breakatwhitespace=true,
  tabsize=3
}


\begin{document}

\maketitle

\section{Wstęp}
Celem projektu było określanie chwil na nagraniu, w których osoba bierze wdech, i w których wydech. 
Dokonano oceny jakościowej za pomocą detekcji oddechu na żywo jak i ilościowej (przy wykorzystaniu
dalej wymienionych metryk.
\section{Podejście}
\subsection{Średnia częstotliwość w czasie}
Pierwotnie przyjętym założeniem było, że podczas wdechu średnia częstotliwość dźwięku jest wyższa niż
gdy osoba wydycha. Z pliku w formacie .wav dla bloków próbek generujemy spektrogramy i obliczamy
średnią ważoną częstotliwość (czyli znacznie redukujemy rozmiar danych tj. $nowy = stary/rozmiar_{bloku}$).
Cechami, na podstawie których
dokonywana byłaby predykcja mogłyby być na przykład $\vec{x} = [\bar{f}(t_0), \bar{f}(t_1), ..., \bar{f}(t_n)]$.
Podejście to jednak jak i pierwotne założenie jest mylne - dla niektórych osób 
dźwięk wydechu jest bowiem wyższy
niż wdechu. Ponadto, stosując to podejście, ograniczylibyśmy się tylko do niektórych typów oddechu (np. 
wdychanie przez nos i wydychanie przez usta).  Biorąc pod uwagę wszystkie wady podejście zostało odrzucone.
\subsection{Dane wejściowe ze spektrogramu}
Innym podejściem jest wzięcie całego spektrogramu (a przynajmniej jego części) jako dane wejściowe do 
modelu.  Podobnie jak ostatnio dzielimy dane na bloki ale tym razem nie liczymy średniej tylko zostawiamy
całą taką klatkę.
Metoda pochodzi od przypuszczenia, że człowiek rozpoznaje i rozróżnia wdech/wydech
na podstawie barwy dźwięku. 

\section{Teoria}
coś tu napisać


\section{Implementacja}
\subsection{Pozyskiwanie częstotliwości i polepszanie jakości danych}
Danymi wejściowymi jest sygnał natężenia dźwięku (np. z mikrofonu, pliku .wav).
Stosujemy FFT dla bloków po 128 próbek każdy, którą następnie uśredniamy i traktujemy jako 
dane wejściowe do modelu (średnia częstotliwość w czasie).\\\\
Do polepszenia jakości danych stosujemy filtr wygładzający Savitzky-Golay z parametrami 50,3.
Filtr ten sekwencyjnie dopasowuje małe podzbiory próbek do wielomianu niskiego stopnia.





\end{document}
