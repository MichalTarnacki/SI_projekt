\documentclass{article}

\title{Projekt z SI}
\author{Dominik Lau, Mateusz Kowalczyk, Michał Tarnacki}

\usepackage{blindtext}
\usepackage{amsmath}
\usepackage[utf8]{inputenc}
\usepackage[polish]{babel}
\usepackage[T1]{fontenc}
\usepackage{listings}
\usepackage{color}
\usepackage{amssymb}
\usepackage{esvect}
\usepackage{graphicx}

\graphicspath{ {./obrazy/} }

\definecolor{dkgreen}{rgb}{0,0.6,0}
\definecolor{gray}{rgb}{0.5,0.5,0.5}
\definecolor{mauve}{rgb}{0.58,0,0.82}

\lstset{frame=tb,
  language=Python,
  aboveskip=3mm,
  belowskip=3mm,
  showstringspaces=false,
  columns=flexible,
  basicstyle={\small\ttfamily},
  numbers=none,
  numberstyle=\tiny\color{gray},
  keywordstyle=\color{blue},
  commentstyle=\color{dkgreen},
  stringstyle=\color{mauve},
  breaklines=true,
  breakatwhitespace=true,
  tabsize=3
}


\begin{document}

\maketitle

\section{Wstęp}
coś tu napisać
\section{Teoria}
coś tu napisać

\section{Implementacja}
\subsection{Pozyskiwanie częstotliwości i polepszanie jakości danych}
Danymi wejściowymi jest sygnał natężenia dźwięku (np. z mikrofonu, pliku .wav).
Stosujemy FFT dla bloków po 128 próbek każdy, którą następnie uśredniamy i traktujemy jako 
dane wejściowe do modelu (średnia częstotliwość w czasie).\\\\
Do polepszenia jakości danych stosujemy filtr wygładzający Savitzky-Golay z parametrami 50,3.
Filtr ten sekwencyjnie dopasowuje małe podzbiory próbek do wielomianu niskiego stopnia.





\end{document}
